\documentclass[a4paper,11pt]{article}
\usepackage[italian]{babel}
\usepackage[utf8]{inputenc}


\begin{document}
\title{Esame di metodi probabilistici}
\date{AA 2009/2010}
\maketitle


%%% \section{Scaletta}

%%% *** scaletta1

%%% - Spazi L2
%%% - Spazi di funzioni e funzioni misurabili
%%% - Probabilità come funzione misurabile (spazio metrico)
%%% - Funzioni generatrice dei momenti
%%% - V.C. con momento secondo stanno in L2
%%% - stimatori
%%% - MVUE geometria
%%% - unicità di mvue

%%% *** scaletta2
%%% - unicità di umvue (geometrica)
%%% - tutta la parte riguardante gli stimatori
%%% - teorema del parallelogramma
%%% - lemma di Riesz (proiezioni)
%%% - propriet1 del valore atteso
%%% - dis. cauchy-schwarz 
%%% - L2 come 



\section{Algebra: gruppi e campi}

sezione da scrivere palrando di gruppi, campi, totalmente ordinato, contabile ecc...



\section{Insiemi e sequenze}
%capitolo della cartella temp

Possiamo cominciare a chiederci perchè per il nostro proposito di utilizzare il concetto di probabilità ci sia bisogno di conoscere la teoria della misura.
Possiamo pensare ad una distribuzione uniforme sui sottoinsiemi dell'intervallo  [0,1], estendendo così il concetto di lunghezza, dato che possiamo definire la misura $P$ tale che

\begin{itemize}
\item $P([a,b)) = b-a \forall 0 \leq a \leq b \leq 1$
\item $P(A_1 \cup A_2 \cup ...)=P(A_1)+P(A_2)+...$ per tutte le sequenze di insiemi disgiunti tali che $A_n \subset [0,1]$ and $n=1,2,...$
% nella riga seguente ci dovrebbe essere un + cerchiato ma non trovo il simbolo adatto quindi uso \otriangle
\item $P(A \otriangle r) = P(A)$ \forall r \in [0,1] dove per A \subset [0,1] abbiamo definito \otriangle come
$$ A\otriangle r = { x \in [0,1] : x-r \in A \vee x-r+1 \in A } $$

\end{itemize)

Non potremo però trovare alcuna funzione P che rispetti quanto detto su tutti i possibili sottoinsiemi dell'intervallo [0,1], quindi ci dovremo concentrare un sottoclassi di insiemi dell'intervallo che chiameremo $\sigma$-algebra.

Per esempio l'insieme dei numeri interi non è un campo dato che non è chiuso rispetto all'inverso del prodotto (ovvero la divisione), mentre lo è l'insieme dei razionali con relative operazioni di somma e prodotto.

\section{Spazi misurabili}

\section{Variabili casuali e funzioni di misura}



\end{document}
