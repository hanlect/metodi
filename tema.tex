\documentclass[a4paper,11pt]{article}
\usepackage[italian]{babel}
\usepackage[utf8]{inputenc}


\begin{document}
\title{Esame di metodi probabilistici}
\date{AA 2009/2010}
\maketitle


%%% \section{Scaletta}

%%% *** scaletta1

%%% - Spazi L2
%%% - Spazi di funzioni e funzioni misurabili
%%% - Probabilità come funzione misurabile (spazio metrico)
%%% - Funzioni generatrice dei momenti
%%% - V.C. con momento secondo stanno in L2
%%% - stimatori
%%% - MVUE geometria
%%% - unicità di mvue

%%% *** scaletta2
%%% - unicità di umvue (geometrica)
%%% - tutta la parte riguardante gli stimatori
%%% - teorema del parallelogramma
%%% - lemma di Riesz (proiezioni)
%%% - propriet1 del valore atteso
%%% - dis. cauchy-schwarz 
%%% - L2 come 





\end{document}
